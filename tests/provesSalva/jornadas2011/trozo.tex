\centerline{%
\begin{tabular}[c]{c}
\begin{tabular}[c]{c@{\protect\rule{\columnsep}{0cm}}c@{\protect\rule{\columnsep}{0cm}}c}
\begin{tabular}{c}
\figuraancho{\anchologo}{Logos/logoUPV.png}
\end{tabular} &
\begin{tabular}[b]{c}
\Huge\textbf{\textcolor{ctextotit}{Adaptacion al grado de la asignatura}}\\\\
\Huge\textbf{\textcolor{ctextotit}{de informática en otras titulaciones}}
\end{tabular} &
\begin{tabular}{c}
\figuraancho{0.7\anchologo}{Logos/logoDSIC} \\
\figuraancho{1.4\anchologo}{Logos/etsid}
\end{tabular}
\end{tabular}\bigskip
\\
\bigskip\bigskip
\Large
\begin{tabular}[b]{c}
\textbf{M.A.~Salido, D.~Segrelles, S.~Espa{\~n}a, D.~Guerrero,
  J.~Martin, P~Ruiz}\\
Departamento de Sistemas Informáticos y Computación \\
Universidad Politécnica de Valencia \\
E-46071 Valencia, SPAIN 
\end{tabular}
\end{tabular}}



El análisis de resultados se plantea en base a los dos cursos docentes
en los que ya se han implantado los grados (2010/2011 –- 2011/2012) en
las diferentes especialidades de la ETSID y los X cursos anteriores en
los que se impartía FIN. El análisis muestra los aspectos positivos y
negativos de la adaptación de la asignatura FIN a INF en términos de
asistencia a clase, alumnos evaluados, alumnos aprobados y coste en
horas por parte del profesor.

%% [1] J. Damian Segrelles, Miguel Angel Salido Gregorio, Adriana Susana
%% Giret Boggin. METODOLOGÍAS ACTIVAS PARA LA ADECUACIÓN DE LA ASIGNATURA
%% FUNDAMENTOS DE INFORMÁTICA AL PLAN BOLÓNIA. Actas del CUIEET 2010
%% (2010).

