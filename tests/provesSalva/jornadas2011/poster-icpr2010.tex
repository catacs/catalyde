\documentclass[portrait]{a0poster}
\setlength{\textwidth}{83cm} % Error en "a0poster": 87cm para esta dimensi�n
                             % en la orientaci�n "portrait"
\usepackage{multicol}
\setlength{\columnsep}{4cm}
\newlength{\anchofiliacion}
\setlength{\anchofiliacion}{39.5cm}

\usepackage{tikz}
\usepackage{calc}
\makeatletter{}
\pgfdeclareshape{corpus}{
  \inheritsavedanchors[from=rectangle]
  \inheritanchor[from=rectangle]{center}
  \inheritanchor[from=rectangle]{west}
  \inheritanchor[from=rectangle]{east}
  \backgroundpath{
     \southwest \pgf@xa=\pgf@x \pgf@ya=\pgf@y
     \northeast \pgf@xb=\pgf@x \pgf@yb=\pgf@y
     \setlength{\pgf@xc}{\pgf@xb-\pgf@xa}
     \setlength{\pgf@yc}{\pgf@yb-\pgf@ya}
     \addtolength{\pgf@yb}{0.4\pgf@yc}
     % Outer border...
     \pgfpathmoveto{\pgfpoint{\pgf@xa}{\pgf@ya}}
     \pgfpatharcaxes{180}{360}{\pgfpoint{0.5\pgf@xc}{0}}{\pgfpoint{0}{0.1\pgf@xc}}
     \pgfpathlineto{\pgfpoint{\pgf@xb}{\pgf@yb}}
     \pgfpatharcaxes{0}{180}{\pgfpoint{0.5\pgf@xc}{0}}{\pgfpoint{0}{0.1\pgf@xc}}
     \pgfpathclose
     % Ellipse completion...
     \pgfpathmoveto{\pgfpoint{\pgf@xa}{\pgf@yb}}
     \pgfpatharcaxes{180}{360}{\pgfpoint{0.5\pgf@xc}{0}}{\pgfpoint{0}{0.1\pgf@xc}}
     % As before, but slightly below...
     \addtolength{\pgf@yb}{-0.1\pgf@yc}
     \pgfpathmoveto{\pgfpoint{\pgf@xa}{\pgf@yb}}
     \pgfpatharcaxes{180}{360}{\pgfpoint{0.5\pgf@xc}{0}}{\pgfpoint{0}{0.1\pgf@xc}}
  }
}
\makeatother{}


\usepackage{amsmath}
\usepackage{latexsym}
\usepackage{pifont}
\usepackage[latin1]{inputenc}
\usepackage{graphicx}
\usepackage{booktabs}
\usepackage{url}
\usepackage{xspace}

\graphicspath{{figures/}}

% Algunos colores...
\definecolor{ctextotit}{rgb}{0.00,0.00,1.00}
\definecolor{cfondoparte}{rgb}{0.80,0.80,1.00}
\definecolor{cbordeparte}{rgb}{0.00,0.00,1.00}
\definecolor{cfondosubparte}{rgb}{0.80,0.80,0.80}
\definecolor{cbordesubparte}{rgb}{0.20,0.20,0.20}
\definecolor{cmarcado}{rgb}{0.90,0.90,0.90}
\definecolor{cnota}{rgb}{0.60,0.60,0.60}
\definecolor{ccarflecha}{rgb}{0.00,0.00,1.00}
\definecolor{cflechaMLP}{rgb}{0.75,0.75,1.00}
\definecolor{cenfasis}{rgb}{1.00,0.00,0.00}
\definecolor{cmarcamiitem}{rgb}{1.00,0.00,0.00}
\definecolor{cmarcaitemlista}{rgb}{0.00,0.00,1.00}
\definecolor{cmarcaitemsublista}{rgb}{0.00,0.00,1.00}

% Para poder conectar nodos en diferentes im�genes TikZ...
\tikzstyle{every picture}+=[remember picture]

% Para marcar una zona de texto con un color y luego poder dibujar l�neas entre zonas marcadas... 
\newsavebox{\cajamarcado}
\newcommand{\nombreMarcado}{}
\newenvironment{marcado}[2]{\renewcommand{\nombreMarcado}{#1}\begin{lrbox}{\cajamarcado}%
\begin{minipage}{#2}}{\end{minipage}%
\end{lrbox}\noindent%
\tikz \node[fill=cmarcado, inner xsep=0pt] (\nombreMarcado) {\usebox{\cajamarcado}};}

% Notas conectadas con una flecha a un punto del texto principal...
\newcommand{\nota}[4][1ex]{\makebox[0pt][l]{%\hspace{0.1em}%
\raisebox{#1}[0pt][0pt]{%
\tikz \draw[<-, cnota, line width=3pt] (0,0)--(#2,#3) node
[shape=rectangle, draw=cnota, right, inner sep=3pt, rounded corners]
{\textrm{\strut\rule{0.5em}{0pt}#4\rule{0.5em}{0pt}}};}}}

% Un car�cter flecha...
\newcommand{\flecha}{\textcolor{ccarflecha}{\ding{233}}}

% Para dar t�tulos a partes del p�ster...
\newcommand{\colapsacentra}[1]{\makebox[0pt]{%
\begin{tabular}[c]{c}
#1
\end{tabular}}}
\newcommand{\parte}[1]{\bigskip%
\centerline{%
\colapsacentra{\tikz \draw [top color=cfondoparte,
                            rounded corners,
                            shading=axis,
                            shading angle=0,
                            draw=cbordeparte] (0,0) rectangle(\columnwidth, 2cm);}%
\colapsacentra{\textbf{\Large#1}}}%
\bigskip}
\newcommand{\subparte}[1]{\medskip%
\centerline{%
\colapsacentra{\tikz \draw [top color=cfondosubparte,
                            rounded corners,
                            shading=axis,
                            shading angle=0,
                            draw=cbordesubparte] (0,0) rectangle(0.75\columnwidth, 1.5cm);}%
\colapsacentra{\textbf{\large#1}}}%
\medskip}

% Incluir una figura dando su anchura o altura...
\newcommand{\figuraancho}[2]{\resizebox{#1}{!}{\includegraphics{#2}}}
\newcommand{\figuraalto}[2]{\resizebox{!}{#1}{\includegraphics{#2}}}
\newcommand{\figuratikzancho}[2]{\resizebox{#1}{!}{\input{TikzFigs/#2}}}
\newcommand{\figuratikzalto}[2]{\resizebox{!}{#1}{\input{TikzFigs/#2}}}

% Marca para �tems fuera de lista...
\newcommand{\miitem}{\noindent\makebox[0pt][r]{\textcolor{cmarcamiitem}{$\rhd$} }\ignorespaces{}}

% Lista con �tems, nivel 1...
\newenvironment{lista}{\begin{list}%
{\textcolor{cmarcaitemlista}{$\bullet$}}%
{\setlength{\leftmargin}{3cm}}%
}{
\end{list}}

% Lista con �tems, nivel 2...
\newenvironment{sublista}{\begin{list}%
{\textcolor{cmarcaitemsublista}{$\circ$}}%
{\setlength{\leftmargin}{3cm}}%
}{
\end{list}}

% Destacar coloreando... 
\renewcommand{\emph}[1]{\textcolor{cenfasis}{#1}}

%%%%%%%%%%%%%%%%%%%%%%%%%%%%%%%%%%%%%%%%%%%%%%%%%%%%%%%%%%%%%%%%%%%%%%%%%%%%%%%
\newcommand{\argmax}{\mathop{\mbox{argmax}}}
\newcommand{\online}{online\xspace}
\newcommand{\offline}{offline\xspace}
\newcommand{\bimodal}{bimodal\xspace}
\newcommand{\State}{\textsc{state}\xspace}

\begin{document}
\thispagestyle{empty}
\sffamily{}\large
%\ \vfill{}

\newlength{\anchologo}
\setlength{\anchologo}{8cm}

\centerline{%
\begin{tabular}[c]{c}
\begin{tabular}[c]{c@{\protect\rule{\columnsep}{0cm}}c@{\protect\rule{\columnsep}{0cm}}c}
\figuraancho{\anchologo}{Logos/logoUPV} &
\begin{tabular}[b]{c}
\Huge\textbf{\textcolor{ctextotit}{Hybrid HMM/ANN models for bimodal online and offline}}\\\\
\Huge\textbf{\textcolor{ctextotit}{cursive word recognition}}
\end{tabular} &
\figuraancho{\anchologo}{Logos/logoDSIC}
\end{tabular}\bigskip
\\
\bigskip\bigskip
\Large
\begin{tabular}[b]{c}
%\begin{tabular}[c]{c@{\protect\rule{\columnsep}{0cm}}c}
%\rule{\anchofiliacion}{0cm} & \rule{\anchofiliacion}{0cm} \\ 
\textbf{S.~Espa{\~n}a-Boquera, J.~Gorbe-Moya,
  F. Zamora-Mart{\'i}nez, M.J. Castro-Bleda}\\
% & \textbf{D. Llorens, A. Marzal, F. Prat, and J. M. Vilar}\\
Departamento de Sistemas Inform�ticos y Computaci�n \\
%& Departament de Llenguatges i Sistemes Inform�tics\\
Universidad Polit�cnica de Valencia 
\\
%& Universitat Jaume I\\
E-46071 Valencia, SPAIN 
%& E-12071 Castell�, SPAIN
\end{tabular}
\end{tabular}}

\vspace{1.5cm}

\begin{multicols}{2}

\parte{The problem}

\miitem Handwriting recognition: \offline
and \online handwriting recognition. \\

\noindent\begin{tabular}{p{18cm}@{\qquad}p{18cm}}
An \offline handwriting
recognition system extracts the information from previously scanned
text images 
&
 ... whereas online systems receive information captured while
the text is being written (stylus and sensitive
tablets).
\\
\begin{center}\includegraphics[scale=0.3]{texto}\end{center}
&
\vspace*{-1.5cm}\begin{center}\includegraphics[scale=5]{imagenonline2}\end{center}
\\

\vspace*{-1.5cm}Offline systems are applicable to a wider range of tasks, given that
\online recognition require the data acquisition to be made with
specific equipment at the time of writing. 
& 
\vspace*{-1.5cm}Online systems are
more reliable due to the additional information available, such as the
order, direction and velocity of the strokes.
\end{tabular}

\medskip

\miitem Recognition performance of current automatic \offline handwriting
transcription systems: far from being perfect. $\to$
Growing interest in assisted transcription systems,
which are more efficient than correcting by hand an automatic
transcription.  
\medskip

\miitem A recent approach to interactive transcription
involves multi-modal recognition, where the user can supply an online
transcription of some of the words: \State{} system.

\medskip

\begin{center}
\includegraphics[scale=0.65]{Wacom}
\includegraphics[scale=0.7]{fotograma2}
\end{center}
\begin{center}
\url{http://state.dlsi.uji.es/state/Home.html}
\end{center}
%\begin{center}\includegraphics[scale=0.8]{fotograma2}\end{center}

\bigskip
\bigskip

\parte{The bimodal contest}

\miitem Description of our \bimodal engine, which entered the
``Bi-modal Handwritten Text Recognition'' contest organized during the
2010 ICPR: Hidden
Markov Models hybridized with neural networks (HMM/ANN) for both
\offline and \online input. 

\miitem $N$-best word hypothesis scores for
both the \offline and the \online samples are combined using a
log-linear combination, achieving very satisfying results.

\bigskip
\bigskip
\parte{The bimodal corpus}

\miitem BiMod-IAM-PRHLT corpus: bimodal dataset of \online and \offline isolated handwritten
words (a vocabulary of 519 words), extracted from IAM corpora.

\miitem Writers of the \online and \offline samples are generally
different.  

\miitem The \offline samples are presented as grey-level images,
and the \online samples are sequences of coordinates describing the
trajectory of an electronic pen. 

\miitem Basic statistics of the biMod-IAM-PRHLT corpus and their
    standard partitions:

\medskip
\begin{center}
\noindent\begin{tabular}{p{9cm}@{\qquad}p{12.5cm}@{\qquad}p{9cm}}
\includegraphics[scale=0.7]{1-original.png} 
\includegraphics[scale=0.7]{a01-020u-09-00-original-House.png}
\includegraphics[scale=0.7]{a06-064-01-00-original-anything.png}
&   
\begin{tabular}{lrr} \toprule
    \multicolumn{1}{l}{Running words} &
    \multicolumn{1}{c}{Online}  &
    \multicolumn{1}{c}{Offline}  
\\\midrule
%Word clases (vocabulary) &	519 &	519\\
Training set&	8\,342 &	14\,409\\
Validation set&	519 &	519\\
(Hidden) test set&	519 &	519\\ \midrule
Total &	9\,380 &	15\,447\\
\bottomrule
\end{tabular}
& 
\includegraphics[scale=0.7]{k10-114z-03_always_3.png}
\includegraphics[scale=0.7]{a01-013z-04_House_18.png}
\includegraphics[scale=0.7]{g10-332z-03_anything_26.png}
\end{tabular}
\end{center}

\phantom{\parte{kkk}\parte{kkk}\parte{kkk}}

%\miitem Examples of word samples from the \bimodal corpus: left, \offline images, and right, \online samples:

%\begin{center}\begin{tabular}{c@{\quad\quad\quad\quad}c}
%\includegraphics[scale=0.8]{1-original.png} &
%\includegraphics[scale=0.8]{k10-114z-03_always_3.png}\\
%\includegraphics[scale=0.8]{a01-020u-09-00-original-House.png} &
%\includegraphics[scale=0.8]{a01-013z-04_House_18.png} \\
%\includegraphics[scale=0.8]{a06-064-01-00-original-anything.png} &
%\includegraphics[scale=0.8]{g10-332z-03_anything_26.png} \\
%\end{tabular}\end{center}


%%%%%%%%%%%%%%%%%%%%%%%%%%%%%%%%%%%%%%%%%%%%%%
%\parte{Offline and online preprocessing}
\parte{Bimodal handwriting recognition engine}

%\miitem Preprocessing example of an \offline sample (left), and an
%  \online sample (right)

\begin{center}
  \begin{tabular}{c@{\quad}c@{\quad}c}
    \includegraphics[scale=0.52]{preproceso-offline.pdf} &
    \includegraphics[scale=0.47]{preproceso-online.pdf} &
  \includegraphics[width=0.45\columnwidth]{figura_hibridos}\\
Offline preprocessing & Online preprocessing & HMM/ANN recognition system
\end{tabular}%\medskip
\end{center}


%%%%%%%%%%%%%%%%%%%%%%%%%%%%%%%%%%%%%%%%%%%%%%
%\parte{Bimodal handwriting recognition engine: HMM/ANN optical models}
%\parte{Bimodal handwriting recognition engine}

%\miitem The recognition system is based on HMM/ANN
%optical models for both \offline and \online recognition:

%\begin{center} \includegraphics[width=0.3\columnwidth]{figura_hibridos}\end{center}

\miitem \textbf{Off-line} HMM/ANN configuration:
\begin{lista}
\item \mbox{7-state} HMM/ANN using a MLP with two
hidden layers of sizes 192 and 128
\item MLP input: current frame plus a context of 4 frames at each side
\item Models trained with the training partition of the
IAM-DB 
\end{lista}

\miitem \textbf{On-line} HMM/ANN configuration:
\begin{lista}
\item Same HMMs topologies and
MLP, but 
\item MLP input wider context:  12 feature frames at both sides
\item Models  trained with the training partition of the
IAM-online DB
\end{lista}


%%%%%%%%%%%%%%%%%%%%%%%%%%%%%%%%%%%%%%%%%%%%%%
%\parte{Bimodal system}
\miitem {Bimodal system}

\begin{enumerate}
\item Scores of the 100 most probable word hypothesis for
the \offline sample using the \offline preprocessing and HMM/ANN
optical models. 
\item Same process  applied to the \online sample.
\item The final score for each \bimodal sample is computed from these lists by means
of a log-linear combination of the scores computed by both the
\offline and \online HMM/ANN classifiers:

\[
\hat{c}=\argmax_{1\le c\le C} ((1-\alpha) \log P(x_{\text{off-line}}|c)
+ \alpha \log P(x_{\text{on-line}}|c))
\]

\item Combination
coefficients  estimated by exhaustive scanning over validation set. 
\end{enumerate}

%\miitem Bimodal recognition engine summary:

\begin{center}
\begin{tabular}{@{~}l@{\qquad}l@{\quad}l@{~}}\toprule
Bimodal system                  & offline   & online    \\ \midrule
\# input features per frame&  60       &    8      \\
\# HMM states     &   7       &    7      \\
MLP hidden layers' size   & 192-128   & 192-128   \\
MLP input context (left-current-right) & 4-1-4     & 12-1-12   \\
Combination coefficient       &   (1-$\alpha$)=0.55    &   $\alpha$=0.45    \\
\bottomrule
\end{tabular}
\end{center}


\parte{Experimental results}




\begin{center}
\begin{tabular}{llcccc} \toprule
\multicolumn{2}{c}{} & \multicolumn{2}{c}{Unimodal} & 	\multicolumn{2}{c}{Bimodal} \\  \midrule
\multicolumn{2}{c}{System} & ~Offline & Online~ & ~Combination &	Relative improv.~\\  \midrule

~Validation   & Baseline & 27.6 & 6.6 & 4.0 & 39\% \\
             & HMM/ANN & 12.7 & 2.9 & 1.9 & 34\% \\  \midrule
~(Hidden) Test& HMM/ANN & 12.7 & 3.7 & 1.5 & 59\% \\ 
\bottomrule
\end{tabular}
\end{center}

\miitem Performance of the \bimodal recognition engine: close to 60\% of improvement is achieved with the \bimodal system when compared to using only the \online system for the test set.



%%%%%%%%%%%%%%%%%%%%%%%%%%%%%%%%%%%%%%%%%%%%%%%%%%%%%%%%%%%%%%%%%%%%%%%%%%
\parte{Conclusions}

\miitem Perfect transcription for most handwriting tasks cannot be achieved: human intervention  needed to correct it $\to$ Assisted transcription
systems aim to minimize  human correction effort. 

\miitem Integration
of \online input into the \offline transcription system can help in
this process (\State{}
system).  

%\miitem This work presents our \bimodal recognition engine and tests it in the ``Bi-modal
%Handwritten Text Recognition'' contest organized during the 2010 ICPR.

\miitem Hybrid HMM/ANN optical models perform very well for both \offline and
\online data, and their naive combination is able to greatly
outperform each system.   

\miitem More exhaustive
experimentation is needed, with a larger corpus, in order to obtain
more representative conclusions.

% As a future work we plan to fully
%integrate it in the assisted transcription system.

%\begin{center}\includegraphics[scale=0.8]{fotograma2}\end{center}


%\parte{Acknowledgments}
\newlength{\altologo}
\setlength{\altologo}{3.5cm}
\newlength{\logosep}
\setlength{\logosep}{1cm}

\begin{marcado}{agradecimientos}{\columnwidth}
\centerline{%
\figuraalto{\altologo}{Logos/logoMiCInn}\qquad SD-TEAM (TIN2008-06856-C05-02)\qquad\qquad\qquad\qquad
}
%\rule{\logosep}{0cm}%
%\figuraalto{\altologo}{Logos/logoMIPRCV}\rule{\logosep}{0cm}%
%\figuraalto{\altologo}{Logos/logoGVA}\rule{\logosep}{0cm}%
%\figuraalto{\altologo}{Logos/logoFCCB}}
\end{marcado}

\end{multicols}
\end{document}

%%% Local Variables: 
%%% mode: latex
%%% ispell-local-dictionary: "american"
%%% TeX-master: t
%%% End: 
